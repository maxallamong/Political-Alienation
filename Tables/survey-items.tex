\subsection{Survey Items}\label{si:survey-items} % survey items
\singlespacing

\begin{itemize}
	\item Age
		\begin{itemize}
			\item Question Wording: ``What is the month, day, and year of your birth?"
			\item Coding: in years, calculated from year of birth
		\end{itemize}
		
%	\item Anti-Immigrant Attitudes
%		\begin{itemize}
%			\item Created by averaging responses to the three question below. Each item re-coded so that higher values = more anti-immigrant sentiment before the three are combined
%			\item Birthright citizenship
%			\begin{itemize}
%				\item Question Wording: ``Some people have proposed that the U.S. Constitution should be changed so that the children of unauthorized immigrants do not automatically get citizenship if they are born in this country. Do you favor, oppose, or neither favor nor oppose this proposal?"
%				\item If R favors or opposes this change, strength of attitude is probed
%				\item Coding: (1) Favor a great deal, (2) Favor a moderate amount, (3) Favor a little, (4) Neither favor nor oppose, (5) Oppose a little, (6) Oppose a moderate amount, (7) Oppose a great deal
%			\end{itemize}
%			\item Childhood arrivals
%			\begin{itemize}
%				\item Question Wording: ``What should happen to immigrants who were brought to the U.S. illegally as children and have lived here for at least 10 years and graduated high school here? Should they be sent back where they came from, or should they be allowed to live and work in the United States?"
%				\item Upon answering the above prompt, strength of attitude if probed
%				\item Coding: (1) Should send back - favor a great deal, (2) Should send back - favor a moderate amount, (3) Should send back - favor a little, (4) Should allow to stay - favor a little, (5) Should allow to stay - favor a moderate amount, (6) Should allow to stay - favor a great deal
%			\end{itemize}
%			\item Build wall with Mexico
%			\begin{itemize}
%				\item Question Wording: ``Do you favor, oppose, or neither favor nor oppose building a wall on the U.S. border with Mexico?
%				\item If R favors or opposes this change, strength of attitude is probed
%				\item Coding: (1) Favor a great deal, (2) Favor a moderate amount, (3) Favor a little, (4) Neither favor nor oppose, (5) Oppose a little, (6) Oppose a moderate amount, (7) Oppose a great deal
%			\end{itemize}
%		\end{itemize}
		
	\item Anti-Trade Attitudes
		\begin{itemize}
			\item Question Wording: ``Do you favor, oppose, or neither favor nor oppose the U.S. making free trade agreements with other countries?" IF  R FAVORS OR OPPOSES, STRENGTH OF ATTITUDE IS PROBED. ``How strongly do you [favor/oppose] it?"
			\item Coding: (1) Favor a great deal, (2) Favor moderately, (3) Favor a little, (4) Neither favor nor oppose, (5) Oppose a little, (6) Oppose moderately, (7) Oppose a great deal
		\end{itemize}
		
	\item Cynicism
	\begin{itemize}
		\item Question Wording:
		\begin{itemize} 
			\item Item 1: ``How often can you trust the federal government in Washington to do what is right?"
			\begin{itemize}
				\item Coding: (1) None of the time/never, (2) Some of the time, (3) Most of the time, (4) Just about always (pre-2016); (1) Always, (2) Most of the time, (3) About half the time, (4) Some of the time, (5) Never (2016-2020)
			\end{itemize}
			\item Item 2: ``Would you say the government is pretty much run by a few big interests looking out for themselves or that it is run for the benefit of all people?"
			\begin{itemize}
				\item Coding: (1) Run by a few big interests, (2) For the benefit of all people
			\end{itemize}
			\item Item 3: ``Do you think that people in government waste a lot of money we pay in taxes, waste some of it, or don't waste very much of it?"
			\begin{itemize}
				\item Coding: (1) Waste a lot, (2) Waste some, (3) Don't waste very much
			\end{itemize}
			\item Item 4: ``How many people running the government are corrupt?"
			\begin{itemize}
				\item Coding: (1) Quite a few; quite a lot, (2) Not many, (3) Hardly any (pre-2016); (1) All, (2) Most, (3) About half, (4) A few, (5) None (2016-2020)
			\end{itemize}
		\end{itemize}
		\item Coding: Responses arranged so that higher values are more cynical answers, combined via PCA (see SM~\ref{si:desc:alienation})
	\end{itemize}
		
	\item Child-Rearing Authoritarianism
		\begin{itemize}
			\item Question Wording: ``Please tell me which one you think is more important for a child to have\ldots"
			\begin{itemize}
				\item Item 1: Independence or \textit{Respect for elders}
				\item Item 2: Curiosity or \textit{Good manners}
				\item Item 3: \textit{Obedience} or Self-reliance
				\item Item 4: Being considerate or \textit{Well-behaved}
			\end{itemize}
			\item Coding: (1) Non-authoritarian trait, (2) Both, (3) Authoritarian trait
			\item Responses to these four statements are averaged to create a single index with higher values representing more authoritarian views. Authoritarian traits are indicated in italics.
		\end{itemize}
		
		
	\item Democratic-Aligned Group Thermometers
		\begin{itemize}
			\item Question Wording: ``I'd like to get your feelings toward some of our political leaders and other people who are in the news these days. I'll read the name of a person and I'd like you to rate that person using something we call the feeling thermometer. Ratings between 50 degrees and 100 degrees mean that you feel favorable and warm toward the person. Ratings between 0 and 50 degrees mean that you don't feel favorable toward the person and that you don't care too much for that person. You would rate the person at the 50 degree mark if you don't feel particularly warm or cold towards the person. If we come to a person whose name you don't recognize, you don't need to rate that person. Just tell me and we'll move on to the next one\ldots How would you rate $\_\_\_\_$ [Blacks/Muslims/LGBT/Hispanics]
			\item Coding: (0) Least favorable attitudes, (100) Most favorable attitudes
			\item Responses to these four feeling thermometer questions are then average to create a single index of attitudes towards Democratic-aligned groups \parencite{mason2021activating}
		\end{itemize}
						
	\item Economic Assessments - Retrospective
		\begin{itemize}
			\item Question Wording: ``Now thinking about the economy in the country as a whole, would you say that over the past year the nation's economy has gotten better, stayed about the same, or gotten worse? [If R answers `gotten better' or `gotten worse'], Much better or somewhat better?/Much worse or somewhat worse?"
			\item Coding: (1) Much better, (2) Somewhat better, (3) About the same, (4) Somewhat worse, (5) Much worse
			\item These items were re-arranged so that the highest value represented beliefs that the economy had gotten a lot better and the lowest value represented beliefs that the economy had gotten a lot worse.
		\end{itemize}
		
	\item Education
		\begin{itemize}
			\item Question Wording: ``What is the highest level of school you have completed or the highest degree you have received?"
			\item Coding: (1) Less than high school, (2) High school diploma, (3) Some college, no bachelors degree, (4) Bachelors or above
		\end{itemize}
		
	\item Evangelical
		\begin{itemize}
			\item Question Wording: ``Which of the following terms describe your religious beliefs?" (2016); ``Do you consider yourself a fundamentalist, an evangelical, both, or neither?" (2020)
			\item Coding: (1) Evangelic, (0) Otherwise
		\end{itemize}
		
	\item Female
		\begin{itemize}
			\item Question Wording: ``What is your gender?" (2016); ``What is your sex?" (2020)
			\item Coding: (1) Female, (0) Otherwise
		\end{itemize}
		
	\item Ideology 
		\begin{itemize}
			\item Question Wording: ``We hear a lot of talk these days about liberals and conservatives. When it comes to politics, do you usually think of yourself as extremely liberal, liberal, slightly liberal, moderate or middle of the road, slightly conservative, extremely conservative, or haven't you thought much about this?"
			\item Coding: (1) Extremely liberal, (2) Liberal, (3) Slightly liberal, (4) Moderate, middle of the road, (5) Slightly conservative, (6) Conservative, (7) Extremely conservative 
		\end{itemize}
		
	\item Inefficacy
	\begin{itemize}
		\item Question Wording: ``For the following statements, please tell me how strongly you agree or disagree:"
		\begin{itemize} 
			\item Item 1: ``Public officials don't care much what people like me think."
			\item Item 2: ``People like me don't have any say about what the government does."
		\end{itemize}
		\item Coding: (1) Agree strongly, (2) Agree somewhat, (3) Neither agree nor disagree, (4) Disagree somewhat, (5) Disagree strongly
		\item Responses arranged so that higher values are more inefficacious answers, combined via PCA (see SM~\ref{si:desc:alienation})
	\end{itemize}
		
	\item Income
		\begin{itemize}
			\item Coding: in quintiles
		\end{itemize}
		
	\item Modern Sexism Index (MSI)
		\begin{itemize}
			\item Question Wording:
			\begin{itemize}
				\item Item 1: ``When women demand equality these days, How often are they are actually seeking special favors?" (Always)
				\begin{itemize}
					\item Coding: (1) Always, (2) Most of the time, (3) About half the time, (4) Some of the time, (5) Never
				\end{itemize}
				\item Item 2: ``Should the news media pay more attention to discrimination against women, less attention, or the same amount of attention they have been paying lately? [If R answers `more attention' or `less attention'], how much more/less attention should media pay to discrimination against women?
				\begin{itemize}
					\item Coding: (1) A great deal more attention, (2) Somewhat more attention, (3) A little more attention, (4) Same amount of attention, (5) A little less attention, (6) Somewhat less attention, (7) A great deal less attention
				\end{itemize}
				\item Item 3: ``When women complain about harassment, how often do they cause more problems than they solve?" (Always)
				\begin{itemize}
					\item Coding: (1) Always, (2) Most of the time, (3) About half the time, (4) Some of the time, (5) Never
				\end{itemize}
			\end{itemize}
			\item Created by additively indexing responses to the three statements above. Each item coded so that higher values = more sexism before the items are combined (more sexist answer in parentheses). 
		\end{itemize}
		
	\item Partisanship  
		\begin{itemize}
			\item Question Wording: ``Generally speaking do you usually think of yourself as a Democrat, a Republican, an independent, or what? " IF R CONSIDERS SELF A DEMOCRAT OR REPUBLICAN. ``Would you call yourself a strong Democrat/Republican or a not very strong Democrat/Republican?" IF R RESPONDENTS INDEPENDENT OR OTHER. ``Do you think of yourself as closer to the Republican Party or to the Democratic Party?"
			\item Coding: Indicators created for Republicans, Democrats, and Independents/Other, with leaners included as partisans
		\end{itemize}
		
	\item Political Interest
		\begin{itemize}
			\item Question Wording: ``Some people don't pay much attention to political campaigns. How about you? Would you say that you have been very much interested, somewhat interested, or not much interested in the political campaigns so far this year?"
			\item Coding: (1) Not much interested, (2) Somewhat interested, (3) Very much interested
		\end{itemize}
		
	\item White 
		\begin{itemize}
			\item Question Wording: ``I am going to read you a list of five race categories. Please choose one or more races that you consider yourself to be \ldots white? black or African-American? American Indian or Alaska Native? Asian? or Native Hawaiian or other Pacific Islander?"
			\item Coding: Indicators created for `white,' `black,' and `other'
		\end{itemize}
\end{itemize}


\subsection{Question Ordering}\label{si:item-ordering} % item ordering
Below are the variables and 
Items below along with their variable names are listed in the order in which they were presented in the 2016 and 2020 ANES Time Series questionnaires, separated by survey period (i.e., pre-election and post-election).

\subsubsection{2016 ANES}
\begin{itemize}
	\item Pre-election
	\begin{itemize}
		\item Weight = V160102, Strata = V160201, Cluster = V160202, Political interest = V161004, Like anything about Trump? = V161074, Ideology = V161126, Retro. Econ. Assessments = V161140x, Party ID = V161158x, Trust = V161215, Big interests = V161216, Waste = V161217, Corrupt = V161218, Evangelical = V161266d, Age = V161267, Education = V161270, Race = V161310x, Female = V161342, Income = V161361x
	\end{itemize}
	\item Post-election
	\begin{itemize}
		\item Vote choice (general) = V162062x, Turnout (general) = V162065x, Therm: LGBT = V162103, Therm: Muslim = V162106, Oppose free trade = V162176x, Don't care = V162215, No say = V162216, Complex = V162217, MSI: Media = V162231x, MSI: Favors = V162232, MSI: Problems = V162233, Child Trait: Independent = V162239, Child Trait : Manners = V162240, Child Trait: Obedient = V162241, Child Trait: Behaved = V162242, Therm : Hispanic = V162311, Therm: Black = V162312
	\end{itemize}
\end{itemize}


\subsubsection{2020 ANES}
\begin{itemize}
	\item Pre-election
	\begin{itemize}
		\item Weight = V200010b, Cluster = V200010c, Strata = V200010d, Political interest = V201006, Like anything about Trump? = V201110, Ideology = V201200, Party ID = V201231x, Trust = V201233, Big interests = V201234, Waste = V201235, Corrupt = V201236, Retro. Econ. Assessments = V201327x, Evangelical = V201459, Age = V201507x, Education = V201511x, Race = V201549x, Female = V201600, Income = V201617x
	\end{itemize}
	\item Post-election
	\begin{itemize}
		\item Turnout (general) = V202109x, Vote choice (general) = V202110x, Therm: LGBT = V202166, Therm: Muslims = V202168, Don't care = V202212, No say = V202213, Complex = V202214, Child Trait: Independent = V202266, Child Trait: Manners = V202267, Child Trait: Obedient = V202268, Child Trait: Behaved = V202269, MSI: Favors = V202291, MSI: Problems = V202292, Oppose free trade = V202361x, Therm: Hispanic = V202479, Therm: Black = V202480
	\end{itemize}
\end{itemize}

\doublespacing